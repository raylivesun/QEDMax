\documentclass[fleqn]{article}

%% Created with wxMaxima 24.02.0

\setlength{\parskip}{\medskipamount}
\setlength{\parindent}{0pt}
\usepackage{iftex}
\ifPDFTeX
  % PDFLaTeX or LaTeX 
  \usepackage[utf8]{inputenc}
  \usepackage[T1]{fontenc}
  \DeclareUnicodeCharacter{00B5}{\ensuremath{\mu}}
\else
  %  XeLaTeX or LuaLaTeX
  \usepackage{fontspec}
\fi
\usepackage{graphicx}
\usepackage{color}
\usepackage[leqno]{amsmath}
\usepackage{ifthen}
\newsavebox{\picturebox}
\newlength{\pictureboxwidth}
\newlength{\pictureboxheight}
\newcommand{\includeimage}[1]{
    \savebox{\picturebox}{\includegraphics{#1}}
    \settoheight{\pictureboxheight}{\usebox{\picturebox}}
    \settowidth{\pictureboxwidth}{\usebox{\picturebox}}
    \ifthenelse{\lengthtest{\pictureboxwidth > .95\linewidth}}
    {
        \includegraphics[width=.95\linewidth,height=.80\textheight,keepaspectratio]{#1}
    }
    {
        \ifthenelse{\lengthtest{\pictureboxheight>.80\textheight}}
        {
            \includegraphics[width=.95\linewidth,height=.80\textheight,keepaspectratio]{#1}
            
        }
        {
            \includegraphics{#1}
        }
    }
}
\newlength{\thislabelwidth}
\DeclareMathOperator{\abs}{abs}

\definecolor{labelcolor}{RGB}{100,0,0}

\begin{document}

\pagebreak{}
{\Huge {\scshape 1.2 Matrizes e Vetores}}
\setcounter{section}{0}
\setcounter{subsection}{0}
\setcounter{figure}{0}

Uma matriz é uma matriz retangular de números. Por isso


\noindent
%%%%%%%%
%% INPUT:
\begin{minipage}[t]{4.000000em}\color{red}\bfseries
(\% i2)	
\end{minipage}
\begin{minipage}[t]{\textwidth}\color{blue}
mtx:\ matrix([1,\ 0,\ 3],\ [-2,\ 4,\ 1]);
\end{minipage}
%%%% OUTPUT:
\[\displaystyle \tag{mtx} 
\begin{pmatrix}1 & 0 & 3\\
\mathop{-}2 & 4 & 1\end{pmatrix}\mbox{}
\]
%%%%%%%%%%%%%%%%


\noindent
%%%%%%%%
%% INPUT:
\begin{minipage}[t]{4.000000em}\color{red}\bfseries
(\% i3)	
\end{minipage}
\begin{minipage}[t]{\textwidth}\color{blue}
mtx1:\ matrix([\ensuremath{\pi},\ 0],[e,\ 1/2],[-1,\ .83],[sqrt(5),\ -4/7]);
\end{minipage}
%%%% OUTPUT:
\[\displaystyle \tag{mtx1} 
\begin{pmatrix}\ensuremath{\pi}  & 0\\
e & \frac{1}{2}\\
\mathop{-}1 & 0.83\\
\sqrt{5} & \mathop{-}\left( \frac{4}{7}\right) \end{pmatrix}\mbox{}
\]
%%%%%%%%%%%%%%%%


\noindent
%%%%%%%%
%% INPUT:
\begin{minipage}[t]{4.000000em}\color{red}\bfseries
(\% i4)	
\end{minipage}
\begin{minipage}[t]{\textwidth}\color{blue}
mtx2:\ matrix([.2,\ -1.6,\ .32]);
\end{minipage}
%%%% OUTPUT:
\[\displaystyle \tag{mtx2} 
\begin{pmatrix}0.2 & \mathop{-}1.6 & 0.32\end{pmatrix}\mbox{}
\]
%%%%%%%%%%%%%%%%


\noindent
%%%%%%%%
%% INPUT:
\begin{minipage}[t]{4.000000em}\color{red}\bfseries
(\% i6)	
\end{minipage}
\begin{minipage}[t]{\textwidth}\color{blue}
mtx3:\ matrix([0],[0]);
\end{minipage}
%%%% OUTPUT:
\[\displaystyle \tag{mtx3} 
\begin{pmatrix}0\\
0\end{pmatrix}\mbox{}
\]
%%%%%%%%%%%%%%%%


\noindent
%%%%%%%%
%% INPUT:
\begin{minipage}[t]{4.000000em}\color{red}\bfseries
(\% i7)	
\end{minipage}
\begin{minipage}[t]{\textwidth}\color{blue}
mtx4:\ matrix([1,\ 3],[-2,\ 5]);
\end{minipage}
%%%% OUTPUT:
\[\displaystyle \tag{mtx4} 
\begin{pmatrix}1 & 3\\
\mathop{-}2 & 5\end{pmatrix}\mbox{}
\]
%%%%%%%%%%%%%%%%
\end{document}
