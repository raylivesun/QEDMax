\documentclass[fleqn]{article}

%% Created with wxMaxima 24.02.0

\setlength{\parskip}{\medskipamount}
\setlength{\parindent}{0pt}
\usepackage{iftex}
\ifPDFTeX
  % PDFLaTeX or LaTeX 
  \usepackage[utf8]{inputenc}
  \usepackage[T1]{fontenc}
  \DeclareUnicodeCharacter{00B5}{\ensuremath{\mu}}
\else
  %  XeLaTeX or LuaLaTeX
  \usepackage{fontspec}
\fi
\usepackage{graphicx}
\usepackage{color}
\usepackage[leqno]{amsmath}
\usepackage{ifthen}
\newsavebox{\picturebox}
\newlength{\pictureboxwidth}
\newlength{\pictureboxheight}
\newcommand{\includeimage}[1]{
    \savebox{\picturebox}{\includegraphics{#1}}
    \settoheight{\pictureboxheight}{\usebox{\picturebox}}
    \settowidth{\pictureboxwidth}{\usebox{\picturebox}}
    \ifthenelse{\lengthtest{\pictureboxwidth > .95\linewidth}}
    {
        \includegraphics[width=.95\linewidth,height=.80\textheight,keepaspectratio]{#1}
    }
    {
        \ifthenelse{\lengthtest{\pictureboxheight>.80\textheight}}
        {
            \includegraphics[width=.95\linewidth,height=.80\textheight,keepaspectratio]{#1}
            
        }
        {
            \includegraphics{#1}
        }
    }
}
\newlength{\thislabelwidth}
\DeclareMathOperator{\abs}{abs}

\definecolor{labelcolor}{RGB}{100,0,0}

\begin{document}
integer um calculor algebra sobre uma soma linear sobre produtos linear da equação compostaobjetos expressivos de sua soma numerica vetores.


\noindent
%%%%%%%%
%% INPUT:
\begin{minipage}[t]{4.000000em}\color{red}\bfseries
(\% i4)	
\end{minipage}
\begin{minipage}[t]{\textwidth}\color{blue}
qed:\ n+sum(1+ai,a1,a2,an)=a1+a2+an;
\end{minipage}
%%%% OUTPUT:
\[\displaystyle \tag{qed} 
n\mathop{+}\left( \ensuremath{\mathrm{ai}}\mathop{+}1\right) \, \left( \ensuremath{\mathrm{an}}\mathop{-}\ensuremath{\mathrm{a2}}\mathop{+}1\right) \mathop{=}\ensuremath{\mathrm{an}}\mathop{+}\ensuremath{\mathrm{a2}}\mathop{+}\ensuremath{\mathrm{a1}}\mbox{}
\]
%%%%%%%%%%%%%%%%
use extandard notação ...


\noindent
%%%%%%%%
%% INPUT:
\begin{minipage}[t]{4.000000em}\color{red}\bfseries
(\% i5)	
\end{minipage}
\begin{minipage}[t]{\textwidth}\color{blue}
qed1:\ n+product(1+ai,a1,a2,an)=a1+a2+an;
\end{minipage}
%%%% OUTPUT:
\[\displaystyle \tag{qed1} 
n\mathop{+}{{\left( \ensuremath{\mathrm{ai}}\mathop{+}1\right) }^{\ensuremath{\mathrm{an}}\mathop{-}\ensuremath{\mathrm{a2}}\mathop{+}1}}\mathop{=}\ensuremath{\mathrm{an}}\mathop{+}\ensuremath{\mathrm{a2}}\mathop{+}\ensuremath{\mathrm{a1}}\mbox{}
\]
%%%%%%%%%%%%%%%%
suma dos vetores lineares e produtos em space linear of calculos lineares algebra em apilidosconstantes para formação de questinamentos em sequences em amortecedores linear de equaçãoobjetivas esfaziando a Q.E.D  das equações linear sobre objectos composivos sobre as equações.


\noindent
%%%%%%%%
%% INPUT:
\begin{minipage}[t]{4.000000em}\color{red}\bfseries
(\% i9)	
\end{minipage}
\begin{minipage}[t]{\textwidth}\color{blue}
eq:\ qed/qed1;
\end{minipage}
%%%% OUTPUT:
\[\displaystyle \tag{eq} 
\frac{n\mathop{+}\left( \ensuremath{\mathrm{ai}}\mathop{+}1\right) \, \left( \ensuremath{\mathrm{an}}\mathop{-}\ensuremath{\mathrm{a2}}\mathop{+}1\right) }{n\mathop{+}{{\left( \ensuremath{\mathrm{ai}}\mathop{+}1\right) }^{\ensuremath{\mathrm{an}}\mathop{-}\ensuremath{\mathrm{a2}}\mathop{+}1}}}\mathop{=}1\mbox{}
\]
%%%%%%%%%%%%%%%%

\end{document}
