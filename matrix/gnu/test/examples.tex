\documentclass[fleqn]{article}

%% Created with wxMaxima 24.02.0

\setlength{\parskip}{\medskipamount}
\setlength{\parindent}{0pt}
\usepackage{iftex}
\ifPDFTeX
  % PDFLaTeX or LaTeX 
  \usepackage[utf8]{inputenc}
  \usepackage[T1]{fontenc}
  \DeclareUnicodeCharacter{00B5}{\ensuremath{\mu}}
\else
  %  XeLaTeX or LuaLaTeX
  \usepackage{fontspec}
\fi
\usepackage{graphicx}
\usepackage{color}
\usepackage[leqno]{amsmath}
\usepackage{ifthen}
\newsavebox{\picturebox}
\newlength{\pictureboxwidth}
\newlength{\pictureboxheight}
\newcommand{\includeimage}[1]{
    \savebox{\picturebox}{\includegraphics{#1}}
    \settoheight{\pictureboxheight}{\usebox{\picturebox}}
    \settowidth{\pictureboxwidth}{\usebox{\picturebox}}
    \ifthenelse{\lengthtest{\pictureboxwidth > .95\linewidth}}
    {
        \includegraphics[width=.95\linewidth,height=.80\textheight,keepaspectratio]{#1}
    }
    {
        \ifthenelse{\lengthtest{\pictureboxheight>.80\textheight}}
        {
            \includegraphics[width=.95\linewidth,height=.80\textheight,keepaspectratio]{#1}
            
        }
        {
            \includegraphics{#1}
        }
    }
}
\newlength{\thislabelwidth}
\DeclareMathOperator{\abs}{abs}

\definecolor{labelcolor}{RGB}{100,0,0}

\begin{document}
são todos exemplos de matrizes. Usamos a notação


\noindent
%%%%%%%%
%% INPUT:
\begin{minipage}[t]{4.000000em}\color{red}\bfseries
(\% i1)	
\end{minipage}
\begin{minipage}[t]{\textwidth}\color{blue}
mtx:\ matrix([a11,a12,a1n],[a21,a22,a2n],[am1,am2,amn]);
\end{minipage}
%%%% OUTPUT:
\[\displaystyle \tag{mtx} 
\begin{pmatrix}\ensuremath{\mathrm{a11}} & \ensuremath{\mathrm{a12}} & \ensuremath{\mathrm{a1n}}\\
\ensuremath{\mathrm{a21}} & \ensuremath{\mathrm{a22}} & \ensuremath{\mathrm{a2n}}\\
\ensuremath{\mathrm{am1}} & \ensuremath{\mathrm{am2}} & \ensuremath{\mathrm{amn}}\end{pmatrix}\mbox{}
\]
%%%%%%%%%%%%%%%%
para uma matriz geral de tamanho m × n (leia-se “m por n”), onde m denota o número de linhas emA e n denota o número de colunas. Assim, os exemplos anteriores de matrizes têmrespectivos tamanhos 2 × 3, 4 × 2, 1 × 3, 2 × 1 e 2 × 2. Uma matriz é quadrada se m = n, ou seja,tem o mesmo número de linhas que colunas. Um vetor coluna é uma matriz m × 1, enquanto uma linhavetor é uma matriz 1 × n. Como veremos, os vetores coluna são de longe os mais importantesdos dois, e o termo “vetor” sem qualificação sempre significará “vetor coluna”.Uma matriz 1 × 1, que possui apenas uma única entrada, é um vetor linha e coluna.O número que está na i-ésima linha e na j-ésima coluna de A é chamado de entrada (i, j)de A, e é denotado por aij. O índice da linha sempre aparece primeiro e o índice da colunasegundo.† Duas matrizes são iguais, A = B, se e somente se elas tiverem o mesmo tamanho, digamos m × n,e todas as suas entradas são iguais: aij = bij para i = 1, . . . , m e j = 1, . . . , n.Um sistema linear geral de m equações em n incógnitas assumirá a forma


\noindent
%%%%%%%%
%% INPUT:
\begin{minipage}[t]{4.000000em}\color{red}\bfseries
 --\ensuremath{\ensuremath{>}}	
\end{minipage}
\begin{minipage}[t]{\textwidth}\color{blue}

\end{minipage}

\noindent%



\noindent
%%%%%%%%
%% INPUT:
\begin{minipage}[t]{4.000000em}\color{red}\bfseries
(\% i3)	
\end{minipage}
\begin{minipage}[t]{\textwidth}\color{blue}
mtx2:\ matrix([a11+x1,a12+x2,a1n+x1n],[a21+x1,a22+x2,a2n+x2n]);
\end{minipage}
%%%% OUTPUT:
\[\displaystyle \tag{mtx2} 
\begin{pmatrix}\ensuremath{\mathrm{x1}}\mathop{+}\ensuremath{\mathrm{a11}} & \ensuremath{\mathrm{x2}}\mathop{+}\ensuremath{\mathrm{a12}} & \ensuremath{\mathrm{x1n}}\mathop{+}\ensuremath{\mathrm{a1n}}\\
\ensuremath{\mathrm{x1}}\mathop{+}\ensuremath{\mathrm{a21}} & \ensuremath{\mathrm{x2}}\mathop{+}\ensuremath{\mathrm{a22}} & \ensuremath{\mathrm{x2n}}\mathop{+}\ensuremath{\mathrm{a2n}}\end{pmatrix}\mbox{}
\]
%%%%%%%%%%%%%%%%
Como tal, é composto por três ingredientes básicos:a matriz de coeficientesm × n A, comentradas aij como em (1.6), o vetor coluna x =


\noindent
%%%%%%%%
%% INPUT:
\begin{minipage}[t]{4.000000em}\color{red}\bfseries
(\% i5)	
\end{minipage}
\begin{minipage}[t]{\textwidth}\color{blue}
mtx3:\ matrix([x1],\ [x2],\ [xn]);
\end{minipage}
%%%% OUTPUT:
\[\displaystyle \tag{mtx3} 
\begin{pmatrix}\ensuremath{\mathrm{x1}}\\
\ensuremath{\mathrm{x2}}\\
\ensuremath{\mathrm{xn}}\end{pmatrix}\mbox{}
\]
%%%%%%%%%%%%%%%%
Observação. Usaremos consistentemente letras minúsculas em negrito para denotar vetores, eletras maiúsculas comuns para denotar matrizes gerais.


\noindent
%%%%%%%%
%% INPUT:
\begin{minipage}[t]{4.000000em}\color{red}\bfseries
(\% i7)	
\end{minipage}
\begin{minipage}[t]{\textwidth}\color{blue}
A\ =\ matrix([2,0,1,3],[-1,2,7,-5],[6,-6,-3,4]);
\end{minipage}
%%%% OUTPUT:
\[\displaystyle \tag{\% o7} 
A\mathop{=}\begin{pmatrix}2 & 0 & 1 & 3\\
\mathop{-}1 & 2 & 7 & \mathop{-}5\\
6 & \mathop{-}6 & \mathop{-}3 & 4\end{pmatrix}\mbox{}
\]
%%%%%%%%%%%%%%%%


\noindent
%%%%%%%%
%% INPUT:
\begin{minipage}[t]{4.000000em}\color{red}\bfseries
 --\ensuremath{\ensuremath{>}}	
\end{minipage}
\begin{minipage}[t]{\textwidth}\color{blue}

\end{minipage}

\noindent%

\end{document}
