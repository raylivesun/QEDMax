\documentclass[fleqn]{article}

%% Created with wxMaxima 24.02.0

\setlength{\parskip}{\medskipamount}
\setlength{\parindent}{0pt}
\usepackage{iftex}
\ifPDFTeX
  % PDFLaTeX or LaTeX 
  \usepackage[utf8]{inputenc}
  \usepackage[T1]{fontenc}
  \DeclareUnicodeCharacter{00B5}{\ensuremath{\mu}}
\else
  %  XeLaTeX or LuaLaTeX
  \usepackage{fontspec}
\fi
\usepackage{graphicx}
\usepackage{color}
\usepackage[leqno]{amsmath}
\usepackage{ifthen}
\newsavebox{\picturebox}
\newlength{\pictureboxwidth}
\newlength{\pictureboxheight}
\newcommand{\includeimage}[1]{
    \savebox{\picturebox}{\includegraphics{#1}}
    \settoheight{\pictureboxheight}{\usebox{\picturebox}}
    \settowidth{\pictureboxwidth}{\usebox{\picturebox}}
    \ifthenelse{\lengthtest{\pictureboxwidth > .95\linewidth}}
    {
        \includegraphics[width=.95\linewidth,height=.80\textheight,keepaspectratio]{#1}
    }
    {
        \ifthenelse{\lengthtest{\pictureboxheight>.80\textheight}}
        {
            \includegraphics[width=.95\linewidth,height=.80\textheight,keepaspectratio]{#1}
            
        }
        {
            \includegraphics{#1}
        }
    }
}
\newlength{\thislabelwidth}
\DeclareMathOperator{\abs}{abs}

\definecolor{labelcolor}{RGB}{100,0,0}

\begin{document}
o apilido da álgebra por expreção linear to movimentos previstos sobre suas matrizes construindoum metodo linear de suas equações compostas de numeros em equações lineares.


\noindent
%%%%%%%%
%% INPUT:
\begin{minipage}[t]{4.000000em}\color{red}\bfseries
(\% i1)	
\end{minipage}
\begin{minipage}[t]{\textwidth}\color{blue}
qed:\ \ x\ +\ y2\ +\ z\ =\ 2;\ 
\end{minipage}
%%%% OUTPUT:
\[\displaystyle \tag{qed} 
z\mathop{+}\ensuremath{\mathrm{y2}}\mathop{+}x\mathop{=}2\mbox{}
\]
%%%%%%%%%%%%%%%%


\noindent
%%%%%%%%
%% INPUT:
\begin{minipage}[t]{4.000000em}\color{red}\bfseries
(\% i2)	
\end{minipage}
\begin{minipage}[t]{\textwidth}\color{blue}
qed1:\ \ x2\ +\ y6\ +\ z\ =\ 7;
\end{minipage}
%%%% OUTPUT:
\[\displaystyle \tag{qed1} 
z\mathop{+}\ensuremath{\mathrm{y6}}\mathop{+}\ensuremath{\mathrm{x2}}\mathop{=}7\mbox{}
\]
%%%%%%%%%%%%%%%%


\noindent
%%%%%%%%
%% INPUT:
\begin{minipage}[t]{4.000000em}\color{red}\bfseries
(\% i3)	
\end{minipage}
\begin{minipage}[t]{\textwidth}\color{blue}
qed2:\ \ x\ +\ y\ +\ z4\ =\ 3;
\end{minipage}
%%%% OUTPUT:
\[\displaystyle \tag{qed2} 
\ensuremath{\mathrm{z4}}\mathop{+}y\mathop{+}x\mathop{=}3\mbox{}
\]
%%%%%%%%%%%%%%%%


\noindent
%%%%%%%%
%% INPUT:
\begin{minipage}[t]{4.000000em}\color{red}\bfseries
(\% i4)	
\end{minipage}
\begin{minipage}[t]{\textwidth}\color{blue}
qed3:\ x\ +\ y2\ +\ z\ =\ 2;
\end{minipage}
%%%% OUTPUT:
\[\displaystyle \tag{qed3} 
z\mathop{+}\ensuremath{\mathrm{y2}}\mathop{+}x\mathop{=}2\mbox{}
\]
%%%%%%%%%%%%%%%%


\noindent
%%%%%%%%
%% INPUT:
\begin{minipage}[t]{4.000000em}\color{red}\bfseries
(\% i15)	
\end{minipage}
\begin{minipage}[t]{\textwidth}\color{blue}
qed4:\ x\ +\ y2\ +\ z\ =\ 2;
\end{minipage}
%%%% OUTPUT:
\[\displaystyle \tag{qed4} 
z\mathop{+}\ensuremath{\mathrm{y2}}\mathop{+}x\mathop{=}2\mbox{}
\]
%%%%%%%%%%%%%%%%


\noindent
%%%%%%%%
%% INPUT:
\begin{minipage}[t]{4.000000em}\color{red}\bfseries
(\% i16)	
\end{minipage}
\begin{minipage}[t]{\textwidth}\color{blue}
qed5:\ y2\ -\ z\ \ =\ 3;
\end{minipage}
%%%% OUTPUT:
\[\displaystyle \tag{qed5} 
\ensuremath{\mathrm{y2}}\mathop{-}z\mathop{=}3\mbox{}
\]
%%%%%%%%%%%%%%%%


\noindent
%%%%%%%%
%% INPUT:
\begin{minipage}[t]{4.000000em}\color{red}\bfseries
(\% i17)	
\end{minipage}
\begin{minipage}[t]{\textwidth}\color{blue}
qed6:\ -y2\ \ +\ x3\ =\ 1;
\end{minipage}
%%%% OUTPUT:
\[\displaystyle \tag{qed6} 
\ensuremath{\mathrm{x3}}\mathop{-}\ensuremath{\mathrm{y2}}\mathop{=}1\mbox{}
\]
%%%%%%%%%%%%%%%%
O sistema equivalente (1.3) já é mais simples que o original (1.1). Observe que osegunda e terceira equações não envolvem x (por design) e, portanto, constituem um sistema de doisequações lineares para duas incógnitas. Além disso, uma vez resolvido este subsistema para ye z, podemos substituir a resposta na primeira equação e precisamos apenas resolver uma únicaequação linear para x.Continuamos desta forma, sendo a próxima fase a eliminação do segundovariável, y, da terceira equação adicionando 21 a segunda equação a ela. O resultado é


\noindent
%%%%%%%%
%% INPUT:
\begin{minipage}[t]{4.000000em}\color{red}\bfseries
(\% i18)	
\end{minipage}
\begin{minipage}[t]{\textwidth}\color{blue}
qed7:\ x\ +\ y2\ +\ z\ =\ 2;
\end{minipage}
%%%% OUTPUT:
\[\displaystyle \tag{qed7} 
z\mathop{+}\ensuremath{\mathrm{y2}}\mathop{+}x\mathop{=}2\mbox{}
\]
%%%%%%%%%%%%%%%%


\noindent
%%%%%%%%
%% INPUT:
\begin{minipage}[t]{4.000000em}\color{red}\bfseries
(\% i19)	
\end{minipage}
\begin{minipage}[t]{\textwidth}\color{blue}
qed8:\ y2\ -\ z\ =\ 3;
\end{minipage}
%%%% OUTPUT:
\[\displaystyle \tag{qed8} 
\ensuremath{\mathrm{y2}}\mathop{-}z\mathop{=}3\mbox{}
\]
%%%%%%%%%%%%%%%%


\noindent
%%%%%%%%
%% INPUT:
\begin{minipage}[t]{4.000000em}\color{red}\bfseries
(\% i20)	
\end{minipage}
\begin{minipage}[t]{\textwidth}\color{blue}
qed9:\ 5\ /\ 2\ \^\ \ z\ =\ 5\ /\ 2;
\end{minipage}
%%%% OUTPUT:
\[\displaystyle \tag{qed9} 
\frac{5}{{{2}^{z}}}\mathop{=}\frac{5}{2}\mbox{}
\]
%%%%%%%%%%%%%%%%
que é o sistema simples que procuramos. Está no que é chamado de forma triangular, o que significaque, enquanto a primeira equação envolve todas as três variáveis, a segunda equação envolve apenasa segunda e terceira variáveis, e a última equação envolve apenas a última variável.†A definição “oficial” de linearidade será adiada
Qualquer sistema triangular pode ser resolvido diretamente pelo método de Back Substi-tução. Como o nome sugere, trabalhamos de trás para frente, resolvendo primeiro a última equação, querequer que z = 1. Substituímos esse resultado de volta na penúltima equação, quetorna-se 2 y - 1 = 3, com solução y = 2. Finalmente substituímos esses dois valores por y ez na primeira equação, que se torna x + 5 = 2, e assim a solução para o triangularsistema (1.4) é


\noindent
%%%%%%%%
%% INPUT:
\begin{minipage}[t]{4.000000em}\color{red}\bfseries
(\% i21)	
\end{minipage}
\begin{minipage}[t]{\textwidth}\color{blue}
x\ =\ -3;
\end{minipage}
%%%% OUTPUT:
\[\displaystyle \tag{\% o21} 
x\mathop{=}\mathop{-}3\mbox{}
\]
%%%%%%%%%%%%%%%%


\noindent
%%%%%%%%
%% INPUT:
\begin{minipage}[t]{4.000000em}\color{red}\bfseries
(\% i22)	
\end{minipage}
\begin{minipage}[t]{\textwidth}\color{blue}
y\ =\ 2;
\end{minipage}
%%%% OUTPUT:
\[\displaystyle \tag{\% o22} 
y\mathop{=}2\mbox{}
\]
%%%%%%%%%%%%%%%%


\noindent
%%%%%%%%
%% INPUT:
\begin{minipage}[t]{4.000000em}\color{red}\bfseries
(\% i23)	
\end{minipage}
\begin{minipage}[t]{\textwidth}\color{blue}
z\ =\ 1;
\end{minipage}
%%%% OUTPUT:
\[\displaystyle \tag{\% o23} 
z\mathop{=}1\mbox{}
\]
%%%%%%%%%%%%%%%%
Além disso, como usamos apenas nossa operação básica de sistema linear para passar de (1.1) parasistema triangular (1.4), esta também é a solução para o sistema original de equações lineares,como você pode verificar. Notamos que o sistema (1.1) tem um único - ou seja, um e únicoum — solução, a saber (1.5).
E isso, salvo algumas pequenas complicações que podem surgir de tempos em tempos, é tudoque existe para o método de Eliminação Gaussiana! É extraordinariamente simples, mas éa importância não pode ser subestimada. Antes de explorar as questões relevantes, será útilreformular nosso método em uma notação matricial mais conveniente.
Exercícios1.1.1. Resolva os seguintes sistemas de equações lineares reduzindo-os à forma triangular e depoisusando substituição reversa.


\noindent
%%%%%%%%
%% INPUT:
\begin{minipage}[t]{4.000000em}\color{red}\bfseries
(\% i24)	
\end{minipage}
\begin{minipage}[t]{\textwidth}\color{blue}
a1:\ x\ -\ y\ =\ 7;
\end{minipage}
%%%% OUTPUT:
\[\displaystyle \tag{a1} 
x\mathop{-}y\mathop{=}7\mbox{}
\]
%%%%%%%%%%%%%%%%


\noindent
%%%%%%%%
%% INPUT:
\begin{minipage}[t]{4.000000em}\color{red}\bfseries
(\% i25)	
\end{minipage}
\begin{minipage}[t]{\textwidth}\color{blue}
a2:\ x\ +\ y2\ =\ 3;
\end{minipage}
%%%% OUTPUT:
\[\displaystyle \tag{a2} 
\ensuremath{\mathrm{y2}}\mathop{+}x\mathop{=}3\mbox{}
\]
%%%%%%%%%%%%%%%%


\noindent
%%%%%%%%
%% INPUT:
\begin{minipage}[t]{4.000000em}\color{red}\bfseries
(\% i26)	
\end{minipage}
\begin{minipage}[t]{\textwidth}\color{blue}
b1:\ u6\ +\ v\ =\ 5;
\end{minipage}
%%%% OUTPUT:
\[\displaystyle \tag{b1} 
v\mathop{+}\ensuremath{\mathrm{u6}}\mathop{=}5\mbox{}
\]
%%%%%%%%%%%%%%%%


\noindent
%%%%%%%%
%% INPUT:
\begin{minipage}[t]{4.000000em}\color{red}\bfseries
(\% i27)	
\end{minipage}
\begin{minipage}[t]{\textwidth}\color{blue}
b2:\ u3\ +\ v2\ =\ 5;
\end{minipage}
%%%% OUTPUT:
\[\displaystyle \tag{b2} 
\ensuremath{\mathrm{v2}}\mathop{+}\ensuremath{\mathrm{u3}}\mathop{=}5\mbox{}
\]
%%%%%%%%%%%%%%%%


\noindent
%%%%%%%%
%% INPUT:
\begin{minipage}[t]{4.000000em}\color{red}\bfseries
(\% i28)	
\end{minipage}
\begin{minipage}[t]{\textwidth}\color{blue}
c1:\ \ p\ -\ q\ -\ r\ =\ 0;
\end{minipage}
%%%% OUTPUT:
\[\displaystyle \tag{c1} 
\mathop{-}r\mathop{-}q\mathop{+}p\mathop{=}0\mbox{}
\]
%%%%%%%%%%%%%%%%


\noindent
%%%%%%%%
%% INPUT:
\begin{minipage}[t]{4.000000em}\color{red}\bfseries
(\% i29)	
\end{minipage}
\begin{minipage}[t]{\textwidth}\color{blue}
c2:\ \ p2\ -\ q\ +\ r3\ =\ 3;
\end{minipage}
%%%% OUTPUT:
\[\displaystyle \tag{c2} 
\ensuremath{\mathrm{r3}}\mathop{-}q\mathop{+}\ensuremath{\mathrm{p2}}\mathop{=}3\mbox{}
\]
%%%%%%%%%%%%%%%%


\noindent
%%%%%%%%
%% INPUT:
\begin{minipage}[t]{4.000000em}\color{red}\bfseries
(\% i30)	
\end{minipage}
\begin{minipage}[t]{\textwidth}\color{blue}
d1:\ u2\ +\ v\ +\ w2\ \ =\ 2;
\end{minipage}
%%%% OUTPUT:
\[\displaystyle \tag{d1} 
\ensuremath{\mathrm{w2}}\mathop{+}v\mathop{+}\ensuremath{\mathrm{u2}}\mathop{=}2\mbox{}
\]
%%%%%%%%%%%%%%%%


\noindent
%%%%%%%%
%% INPUT:
\begin{minipage}[t]{4.000000em}\color{red}\bfseries
(\% i31)	
\end{minipage}
\begin{minipage}[t]{\textwidth}\color{blue}
d2:\ -u\ -v\ +\ w3\ =\ 1;
\end{minipage}
%%%% OUTPUT:
\[\displaystyle \tag{d2} 
\ensuremath{\mathrm{w3}}\mathop{-}v\mathop{-}u\mathop{=}1\mbox{}
\]
%%%%%%%%%%%%%%%%
\end{document}
